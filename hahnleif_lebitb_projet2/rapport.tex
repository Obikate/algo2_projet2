\documentclass[a4paper,11pt]{article}

\usepackage[utf8]{inputenc}
%% \usepackage[latin1]{inputenc}
\usepackage[T1]{fontenc}
\usepackage[francais]{babel}
\usepackage{listings}
\usepackage{graphicx}
\usepackage{array}

\title{Files de priorités, Arbres de Huffman}
\author{Hahnlein Felix, Lebit Benjamin}
\date{Mai 2015}

\begin{document}

\maketitle

\section{Avancement du projet et structure de l'archive}
Le code final fourni implémente toutes les optimisations décritent dans le sujet excepté la dernière (découpage des fichiers textes originaux en fichiers tampons pour rendre l'arbre de Huffman plus compact). \\\\
L'archive du projet est constituée de 4 dossiers principaux et d'un Makefile dont nous détaillerons l'utilisation ci-après :
\begin{itemize}
\item Le dossier \textit{fichiers\_compresses} contenant les versions compressées de chaque fichier texte de test du dossier \textit{tests}.
\item Le dossier \textit{resultats} contenant, aprés compilation, les fichiers de test décompressés et un fichier \textit{resultats.txt}, généré par le script du fichier\textit{ script\_tests.sh}, contenant des données d'expérimentation des différents fichiers textes (ces résultats seront explicités dans la section 3)
\item Le dossier \textit{src} contenant le code source. Les fichiers \textit{arbre\_huffman.adb/ads} et \textit{huffman.adb} ont 3 versions en fonction de l'état du code (sans optimmisation, avec décodage code ou version finale).
\item le dossier \textit{tests} contenant tous les fichiers textes au format .txt utilisés pour les tests.
\end{itemize}
La compilation de l'archive se fera en lançant la commande \scriptsize{\textbf{\$make all}} \normalsize{ et la décompilation avec la commande }\scriptsize{\textbf{\$make clean}}. \\
\section{Détails de l'implémentation}

\section{Tests de validation et d'expérimentation} 

\large Résultats en fonction des textes : \\
\begin{tabular}{|l|l|l|l|l|l}
\hline
 & texte1 & texte2 & texte3 & shakespeare \\
\hline
Entropie & 4.15 & 4.17 & 2.25 & 4.66 \\
Compacité & 4.18 & 4.21 & 2.33 & 4.69 \\
Efficacité & 0.92 & 0.92 & 0.97 & 0.94 \\
Taux de compression (stockage du tableau de fréquences) & 1.12 & 0.66 & 172 & 0.59 \\
Taux de compression (stockage de l'arbre) & 0.55  & 0.53  & 2.83  &   & \\
Temps d'exécution(en seconde) & 0.00 & 0.00 & 0.00 & 1.24s \\
\hline
\end{tabular}
\end{document}
