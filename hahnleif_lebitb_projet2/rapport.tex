\documentclass[a4paper,11pt]{article}

\usepackage[utf8]{inputenc}
%% \usepackage[latin1]{inputenc}
\usepackage[T1]{fontenc}
\usepackage[francais]{babel}
\usepackage{listings}
\usepackage{graphicx}
\usepackage{array}

\title{Files de priorités, Arbres de Huffman}
\author{Hähnlein Felix, Lebit Benjamin}
\date{Mai 2015}

\begin{document}

\maketitle

\section{Avancement du projet et structure de l'archive}
Le code final fourni implémente toutes les optimisations décritent dans le sujet excepté la dernière (découpage des fichiers textes originaux en fichiers tampons pour rendre l'arbre de Huffman plus compact). \\\\
L'archive du projet est constituée de 4 dossiers principaux et d'un Makefile dont nous détaillerons l'utilisation ci-après :
\begin{itemize}
\item Le dossier \textit{fichiers\_compresses} contenant les versions compressées de chaque fichier texte de test du dossier \textit{tests}.
\item Le dossier \textit{resultats} contenant, aprés compilation, les fichiers de test décompressés et un fichier \textit{resultats.txt}, généré par le script du fichier\textit{ script\_tests.sh}, contenant des données d'expérimentation des différents fichiers textes (ces résultats seront explicités dans la section 3)
\item Le dossier \textit{src} contenant le code source. Les fichiers \verb+arbre_huffman.adb/ads+ et \verb+huffman.adb+ ont 3 versions en fonction de l'état du code (sans optimmisation, avec décodage code ou version finale).
\item le dossier \textit{tests} contenant tous les fichiers textes au format .txt utilisés pour les tests.
\end{itemize}
La compilation de l'archive se fera en lançant la commande \verb+$make all+ et la décompilation avec la commande \verb+$make clean+. \\
\section{Détails de l'implémentation}
	La structure générale du compresseur était déjà imposée, alors on va juste expliquer les choix qu'on a fait lors de l'implémentation.
	\begin{enumerate}
		\item{\textbf{Style de codage}}\\
			Nous avons volontairement choisi d'utiliser un style de codage différent par rapport à celui du code déjà présent (p.ex. l'utilisation des caractères majuscules) pour permettre au correcteur de bien distinguer quelles parties nous avons implémentées.
		\item{\textbf{File de priorité}}\\
			Nous avons opté pour une représentation par un tas en utilisant un type enregistrement. \\
			Tout au long de son utilisation, nous avons besoin du nombre courant d'éléments qui se trouvent dans la file. On peut donc soit utiliser une variable globale définie dans le fichier \verb+file_priorite.adb+ soit déclarer la file comme un type enregistrement. Pour ne pas avoir des effets de bords, nous avons choisi un type enregistrement.\\
			L'implémentation des fonctions gérant la file est classique, tout en respectant qu'ici nous avons besoin de l'élément le moins prioritaire en "haut" du tas. Il s'agit donc d'une file de priorité "inversée".
		\item{\textbf{Affichage de l'arbre}}\\
			En pratique, l'arbre de Huffman sera trop encombrant pour bien l'afficher sur la sortie standard. On a donc choisi un affichage rudimentaire qui représente les n\oe{}uds sous forme de $0$ et dès qu'on tombe sur une feuille, il affiche le caractère correspondant. Cet affichage ne sert qu'au débogage.
		\item{\textbf{Calcul de l'arbre}}\\
			On a utilisé l'algorithme classique pour la construction récursive de l'arbre de Huffman, i.e. on fusionne à chaque itération les deux n\oe{}uds qui ont la probabilité la plus petite jusqu'à ce qu'il ne reste qu'un seul n\oe{}ud.
		\item{\textbf{Calcul du dictionnaire}}\\
			Le code associé à un caractère s'obtient en parcourant l'arbre et en se souvenant à chaque branchement de la direction (gauche ou droite) choisie.\\
			Ici, on a rencontré l'obstacle suivant: on ne connait pas à l'avance la longueur du code. Or, le code sera stocké dans un tableau dont la déclaration nécessite cette information.\\
			Pour contourner ce problème, on pourrait effectuer une recherche d'élément dans l'arbre. Mais comme il n'existe pas de relation d'ordre dans l'arbre de Huffman, celle-ci serait en $\mathcal{O}(n)$, en notant n le nombre de feuilles de l'arbre.
			Nous avons donc choisi d'utiliser un pointeur vers notre tableau, de déclarer à chaque étape un nouveau tableau, de transmettre les anciennes informations et de libérer l'ancien tableau. La taille du nouveau tableau est incrémenté de $1$. 
			Nous sommes bien conscients que cette solution est également couteuse en temps, car le transfert de l'ancien vers le nouveau tableau est en pire cas en $\mathcal{O}(h)$ où h est la hauteur de l'arbre. Cette considération fait que le calcul d'un seul code est en $\mathcal{O}(h^2)$. 
			Ainsi, la fonction \verb+Calcul_Dictionnaire+ est en $\mathcal{O}(nh^2)$.
		\item{\textbf{Amélioration du décodage de code}}\\
			Le problème de cette procédure était la réallocation et libération du \verb+TabBits+ à chaque étape. On a contourné ce problème en ajoutant une variable globale qui se rappelle de la dernière position dans le tableau désigné par la variable \verb+Reste+. Ainsi, quand on rappelle la fonction \verb+Decodage_Code+, on peut utiliser un tableau auxiliaire unique en parcourant l'arbre.
		\item{\textbf{Encodage de l'arbre}}\\
			Nous avons choisi d'encoder l'arbre comme il a été suggéré par l'énoncé, i.e. en représentant les n\oe{}uds par un $0$ et les feuilles par un $1$. L'implémentation n'est pas compliquée en soi grâce à l'exemple de la fonction \verb+Decodage_Code+ qui effectue en quelque sorte l'opération inverse.\\
			En plus de l'arbre, on sauvegarde aussi le nombre de feuilles dans le fichier compressé pour pourvoir le décompresser.
	\end{enumerate}

\section{Tests de validation et d'expérimentation} 
\begin{itemize}
\item Le texte \verb+texte1.txt+ est un texte aléatoire généré par Lorem Ipsum de 3 paragraphes de taille 1725 octects.
\item Le texte \verb+texte2.txt+ est un texte aléatoire généré par Lorem Ipsum de 20 paragraphes de taille 7522 octects.
\item Le texte \verb+texte3.txt+ est la chaine de caractères `hello' de taille 6 octects.
\item Le texte \verb+shakespeare.txt+ est l'ensemble des oeuvres de William Shakespeare de taille 5589917 octects.
\end{itemize}
Le choix des exemples à été fait de manière à avoir des textes de tailles différentes pour pouvoir comparer les résultats.\\\\
\vspace{5mm}
\textbf{ Résultats en fonction des textes :} \\\\
\begin{tabular}{|l|c|c|c|r|}
\hline
& texte1 & texte2 & texte3 & shakespeare \\
\hline
Entropie & 4.15 & 4.17 & 2.25 & 4.66 \\
\hline
Compacité & 4.18 & 4.21 & 2.33 & 4.69 \\
\hline
Efficacité & 0.92 & 0.92 & 0.97 & 0.94 \\
\hline
Taux de compression & & & & \\
(stockage du tableau de fréquences) & 1.12 & 0.66 & 172 & 0.59 \\
\hline
Taux de compression & & & & \\
(stockage de l'arbre) & 0.55  & 0.53  & 2.83  &  0.59\\
\hline
Temps d'exécution & & & &\\
(en seconde) & 0.00 & 0.00 & 0.00 & 1.24 \\
\hline
\end{tabular}
\\\\
Le test de validation pour la décompression peut s'effectuer en tapant la commande :\\
\verb+diff tests/<nom_fichier_texte>+\\
\verb+ resultats/<dossier_version_code>/<nom_fichier_texte>+ pour vérifier que l'on a bien 2 fichiers identiques avant et après décompression.
\subsection{Interprétaion des résultats}
\end{document}
